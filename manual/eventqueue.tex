\section{Event Queue}
\label{sec:eventQueue}


\subsection{Operation}
\label{ss:eventQueueOperation}

An AOI state machine is considered in a stable state configuration when
not executing a transition. Every time it is scanned true in this stable
condition it will evaluate received events to determine
if a transition is triggered
originating from the currently active state configuration. Trigger evaluation
is applied to events one at a time, and if a transition is triggered,
further evaluation of events is postponed until the active transition
fully completes and returns the state machine to another stable state
configuration. This process of handling events one a time and deferring
events while executing a transition is referred to in
\textcite[\S14.2.3.9.1]{OMGUML} as the
\introTechnicalTerm{run-to-completion paradigm}.

The arrival of events in the form of event input rising edges can
temporarily exceed the state machine's evaluation ability since the latter
will only consume events while stable, yet event inputs have no such
restriction. This situation can occur when multiple events arrive in the
same scan or during a transition, thus the state machine cannot immediately
process some or all of the new events.
To ensure events are not lost in these cases
a storage mechanism is required to hold events until
the state machine reaches a stable state configuration.
\textcite[\S13.3.3]{OMGUML} describes an
\introTechnicalTerm{event pool} where events are stored until dispatched,
one at a time, to the state machine. LogixUML implements this event pool
as an \introTechnicalTerm{event queue} using first-in first-out, or
\introTechnicalTerm{FIFO}, behavior, where the order events are
placed into the queue is the same in which they are removed and dispatched
to the state machine. Events are queued in scan order, i.e., new events are
always queued after any arriving in previous scans. The order of
events arriving in the same scan is undefined.

The basic outline of an AOI state machine's execution during each true scan,
implementing the run-to-completion paradigm with the event queue, is
illustrated in Figure~\ref{fig:aoiFlowchart}.

\begin{description}
  \item[Enqueue Events] All event inputs are evaluated to detect rising
    edges, which cause their associated events to be placed into the
    event queue.

  \item[Stable] Is the state machine in a stable configuration? In other
    words, is it not currently in the midst of a transition?

  \item[Advance Transition] Proceed to the next step in the current
    transition according to the state machine's configured scan mode.
    AOI execution then ends unconditionally, even if the next step is the
    final step in the transition: the stable configuration of the
    target state.

  \item[Queue Empty] Evaluate the content of the event queue to see if it
    holds any events. If no events are found, AOI execution ends and the
    state machine remains stable.

  \item[Dequeue Event] Remove a single event from the event queue according
    to FIFO ordering.

  \item[Transition Triggered] Evaluate the dequeued event to determine if it
    satisfies a transition trigger from the current state configuration.
    If a transition is triggered, AOI execution completes with the
    first step of the transition, otherwise the event is discarded and
    execution loops back to fetch the next event.
\end{description}

\begin{figure}
  \centering
  \begin{tikzpicture}

    \node (start)
          [flowchart terminator]
          {Start};

    \node (enqueue)
          [flowchart process, below=of start]
          {Enqueue\\Events};

    \node (stable)
          [flowchart decision, below=of enqueue]
          {Stable};

    \node (queue empty)
          [flowchart decision, below left=of stable]
          {Queue\\Empty};

    \node (dequeue)
          [flowchart process, below=of queue empty]
          {Dequeue\\Event};

    \node (check trigger)
          [flowchart decision, left=of dequeue]
          {Transition\\Triggered};

    \node (advance transition)
          [flowchart process, below right=of stable]
          {Advance\\Transition};

    \node (end)
          [flowchart terminator, below=6 of stable]
          {End};


    \begin{scope}[flowchart arrow, every node/.style=very near start]
      \draw (start) -- (enqueue);

      \draw (enqueue) -- (stable);

      \draw (stable) -| node[anchor=south] {No} (advance transition);
      \draw (stable) -| node[anchor=south] {Yes} (queue empty);

      \draw (queue empty) -- node[anchor=east] {No} (dequeue);
      \draw (queue empty) -| node[anchor=south] {Yes} (end);

      \draw (dequeue) -- (check trigger);

      \draw (check trigger) |- node[anchor=east] {No} (queue empty);
      \draw (check trigger) |- node[anchor=east] {Yes} (end);

      \draw (advance transition) |- (end);
    \end{scope}

  \end{tikzpicture}
  \caption{Add-on instruction logic flow.}
  \label{fig:aoiFlowchart}
\end{figure}



\subsection{Configuration}

The event queue capacity is configured with the
\identifier{eventQueueSize} property of the
\identifier{stateMachineAoi} stereotype. Valid values are~1--8,
inclusive, specifying the maximum number of events that can be queued.
When selecting a size keep in mind every event is stored in
the event queue before being dispatched to the state machine,
even when the event queue is initially empty and a new event can be
evaluated immediately.


\subsection{Overflow}

Overflow of the event queue will occur if an event arrives when the queue
is full, and this condition must be handled as part of the guarantee that
events are not lost.
LogixUML considers event queue overflow a design problem,
with the state machine or the surrounding application logic, and
therefore treats overflow as an unrecoverable error by stopping the processor.
Additionally, the AOI will energize its \identifier{eventQ\_overflow}
output to help identify the culprit state machine instance; this output
will automatically reset during prescan. Internally, the AOI generates
a major fault to halt the processor, which can also be used to locate the
problem state machine. Do not use the controller fault handler
to clear the major fault caused by an event queue overflow.


\subsection{Monitoring}

State machine add-on instructions internally monitor the available space in
the event queue to maintain a record of peak usage. This metric is
available through the \identifier{eventQ\_watermark} output
to evaluate both event queue sizing and state machine performance
\emph{in situ}. The output functions as a low
watermark, reflecting the smallest number of free slots encountered during
operation; a value of zero indicates the event queue was completely
full at some point.

The watermark function can be cleared with the
\identifier{eventQ\_watermark\_reset} input. Setting this input true will
reset the watermark to current event queue capacity, overwriting any
previously captured value. Resetting the watermark has no effect on event
queue operation; it does not cause any events to be discarded or inhibit
queuing of new events. The watermark input can be exercised at any
time without impacting state machine behavior.

The reset function is available only when the
AOI is evaluated true or false during normal program scan; it does not
work in prescan. The reset input is level sensitive, i.e., the reset
function persists as long as the input remains true.


\subsection{Examples}

Operation of the run-to-completion paradigm and event queue is
demonstrated using the state machine depicted in
Figure~\ref{fig:eventQueueStateMachine}. For simpliticy these examples
assume the state machine is configured for single transition scan mode
as described in \S\ref{ss:scanModeSingle},
and begin with the state machine stable in \identifier{s1}.

\plantUmlFigure{eventQueueStateMachine}
               {State machine for event queue examples.}

In the first example case, shown in Figure~\ref{fig:eventQueueTiming1},
events \identifier{e1} and \identifier{e2} arrive simultaneously at scan~1.
As described in \S\ref{ss:eventQueueOperation}, events arriving in the same
scan can be queued in any order, and case~1 explores behavior if
\identifier{e1} is queued before \identifier{e2}. Both events are queued
upon arrival,
and since the state machine is stable in \identifier{s1}, \identifier{e1}
is dequeued and dispatched to the state machine, triggering the transition
to \identifier{s2}. The transition progresses through scans~1 and~2.
The state machine is again stable in \identifier{s2} when the AOI is
evaluated at scan~3, and the next event is removed from the event queue,
\identifier{e2}, triggering the final transition to \identifier{s3}.

\plantUmlFigure{eventQueueTiming1}
               {Event queue timing example, case~1.}

The other possibile result of \identifier{e1} and \identifier{e2} arriving
simultaneously is \identifier{e2} being queued ahead of \identifier{e1};
this is shown in Figure~\ref{fig:eventQueueTiming2}. Both events again
arrive at scan~1, however, this time \identifier{e2} is dequeued first
and dispatched to the state machine, initiating a transition to \identifier{e3}.
The state machine reaches a stable configuration in scan~3 and \identifier{e1}
is then removed from the queue. Event \identifier{e1} is discarded as no
transition triggered by \identifier{e1} exists from \identifier{s3}.

\plantUmlFigure{eventQueueTiming2}
               {Event queue timing example, case~2.}
