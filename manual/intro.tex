\section{Introduction}


\subsection{Overview}

Unified Modeling Language, or UML, is a popular modeling framework for designing
software programs through a set of graphical diagrams defining
the software's requirements, interfaces, structures, and functions.
Although UML is typically utilized with traditional,
general-purpose programming languages, such as Java or C++, UML itself
is language-agnostic and can be very useful for PLC applications written in
ladder logic.

Programmable logic controllers are almost exclusively employed for
event-driven systems, where inputs generate events and the controller responds
by actuating outputs. Typical examples of inputs and outputs are sensors,
operator buttons, valves, and motors. Event-driven systems are often
implemented in PLC software as a sequencer, drum timer, state machine,
or other similarly-named construct where the system responds to events by
progressing through a series of steps that define output conditions.
This document focuses only on state machines as that is the only approach
supported by LogixUML.

The UML state machine diagram provides a formal, well-defined notation for
designing event-driven software, where system behavior is depicted in a
high-level, graphical manner that is generally easier to analyze and understand
than the resulting source code. Implementing the state machine in the
target programming language can be done manually, however, doing so tedious,
error-prone, and almost inevitably leads to situations where the software
is updated but the state machine design diagram is not. Converting the
state machine to source code can also be done automatically via software, and 
several programs exist that can do this for tranditional programming
languages; the intent of LogixUML is to bring this automated code generation
to Rockwell Automation ControlLogix PLCs, allowing UML state machine diagrams
to be automatically converted into ladder logic.


\subsection{Implementation}

LogixUML is not a stand-alone program, but an extension to the Modelio
UML modeling software that handles the details of generating a ladder
logic implementation of a state machine defined by a UML diagram.
Modelio and LogixUML are also not replacements for RSLogix~5000;
after the state machine components are created by LogixUML, RSLogix~5000
is still required to import, integrate, download, debug, and perform all
the other activities associated with PLC software development.

State machine logic generated by LogixUML takes the form of an add-on
instruction, or AOI, which have the following advantages over other
types of logic components:

\begin{itemize}
  \item Add-on instructions are a type of program component that can be
    easily imported by RSLogix~5000 using the L5X file format.

  \item The internal state machine implementation within an add-on
    instruction is completely isolated from the surrounding program,
    so importing state machine AOIs will not impact other tags or logic.

  \item AOIs can be instantiated multiple times for applications that require
    independent instances of the same state machine.
\end{itemize}

The add-on instruction created from a state machine is always complete
in that the entire behavior defined by the state machine,
including all events, transitions, and states, is fully contained by that
single AOI. Furthermore, the exported add-on instructions always implement
exactly one state machine, i.e., there is a one-to-one relationship between a
state machine and the resulting AOI, regardless of how many state machines
are exported by LogixUML.


\subsection{System Requirements}

\begin{itemize}
  \item Modelio, version~4.x.x.

  \item RSLogix~5000 or Studio~5000 Logix Designer, version~16 or later.
\end{itemize}


\subsection{Installation}

\begin{enumerate}
  \item Download and install the Modelio software, available from the
    \textcite{MODELIO}.

  \item Download The LogixUML module from the \textcite{MODELIOSTORE} or
    \textcite{REPO}.

  \item Load the module into Modelio module catalog as described in
    \textcite[Macros catalog]{MODELIOMANUAL}.

  \item Create a project to house the source state machine diagrams;
    the diagrams do not need to be created at this point, just the project.
    See the \textcite[Creating a project]{MODELIOMANUAL} for details.

  \item Deploy the LogixUML module in the Modelio project, which is covered
    in the \textcite[Configuring project modules]{MODELIOMANUAL}.
    This step must be done separately for each project containing state
    machines to be exported as add-on instructions.
\end{enumerate}


\subsection{Workflow}
