\section{Supported UML}

Modelio supports a wide range of elements defined in \textcite{OMGUML}.
LogixUML, however, is only capable of converting a subset of these into
add-on instructions. This section describes the UML elements supported
by LogixUML, along with details on how they are implemented in the output
AOIs.

\subsection{State Machines}

State machines are the top-level diagram type defining behavior
LogixUML implements as an add-on instruction, of which \textcite[\S14]{OMGUML}
defines two kinds: \introTechnicalTerm{behavior}
and \introTechnicalTerm{protocol}. Only behavior state machines are
supported, and are configured as such by setting the state machine
\identifier{Kind} property to \identifier{Dynamic}.

Any state machine intended to be exported as an add-on instruction
must have the \identifier{\stereotypeName} stereotype applied.
This stereotype is defined by the LogixUML module, and serves to
identify the state machine as a canididate for exporting to an add-on
instruction and provide configuration properties customizing the
AOI's implementation. Details about stereotype properties are presented
in~\S\ref{sec:scanMode} and~\S\ref{sec:eventQueue}.

The name of the resulting add-on instruction will be the state machine's
name prefixed with \identifier{stateMachine\_}, e.g., exporting a state
machine named \identifier{monitor} will yield a
\identifier{stateMachine\_monitor} add-on instruction.
The state machine name must therefore conform to the convention defined
in \textcite[Chapter~2]{AOI}.

In addition to the parameters defined by the state machine's content,
such as states and events, the add-on instruction will include a
boolean output representing the condition of the entire state machine.
This output, named \identifier{active}, is unconditionally set to~1 when
the AOI is scanned true, and set to~0 in Prescan or if the AOI
is scanned false. The \identifier{active} output is defined as visible
and non-required, appearing as a flag on the right side of the
instruction's face.

Behavior of the state machine add-on instruction for the various scan
modes is as follows.

\begin{description}
  \item[Prescan] All outputs are set to zero. Any information regarding
    the active state or transition is reset. The event queue, which is
    covered in~\S\ref{??}, is emptied.

  \item[True] The rung-out condition and \identifier{active} output are set
    true. Event inputs are evaluated to detect rising edges. The state machine
    operates starting with its initial transition. The entry, exit, and
    do outputs are energized according to the state machine's condition.

  \item[False] Same as Prescan with the addition that the state machine
    remains inert, not evaluating or responding to any event inputs.
\end{description}


\subsection{States}
\label{ss:states}

LogixUML supports \introTechnicalTerm{simple} and
\introTechnicalTerm{composite} states as defined in
\textcite[\S14.2.3.4.1]{OMGUML}; \introTechnicalTerm{submachine} states are
not supported. Composite states can be nested to any depth, however, a
composite state may only have a single region.

Every state must have a name unique throughout the entire state
machine. State names are used to construct add-on instruction parameter names,
so they are not case-sensitive and must confirm to Logix tag name
conventions.

The exported add-on instruction will include a set of boolean outputs
for each state, representing entry, exit, and do activities described
in \textcite[\S14.2.3.4.3]{OMGUML}, and are intended to
be used by the application logic to enable actions associated with each
state. The outputs are named with the prefixes
\identifier{stateEntry\_}, \identifier{stateExit\_}, and
\identifier{stateDo\_}, followed by the state name. These outputs are defined
as neither visible nor required, so they are not shown on the add-on
instruction's face and do not need to be assigned a separate tag.

One additional boolean output is provided for each state, named with the
\identifier{stateActive\_} prefix. It is simply the logical OR of the
entry, exit, and do outputs. This output is defined as visible and not
required, so it will appear as a flag on the right edge of the
instruction face.


\subsection{Initial Pseudostates}

Initial pseudostates are supported in two contexts: at the state machine's
top-level region, and within a composite state. The following requirements
apply to both:

\begin{itemize}
  \item There can be at most one initial pseudostate in a region.

  \item An initial pseudostate must have exactly one outgoing transition.

  \item The outgoing transition must not have a triggering event.
\end{itemize}

A state machine must have an initial pseudostate defining its behaviour
immediately after the AOI is scanned true; the departing transition may lead
to any state. Initial pseudostates are optional for composite states,
and if present, must terminate on a substate directly or indirectly contained
by the region owning the initial pseudostate.

Figure~\ref{fig:initialPseudostateCompositeStates} contains examples of
initial pseudostates in composite states. State \identifier{superX}
contains a valid initial pseudostate with a transition leading to
a child substate; \identifier{subY} would also be a valid target
for this transition. The initial pseudostate in \identifier{s1} is
ill-formed because the target state, \identifier{s2}, is not contained within
the same region.

\plantUmlFigure{initialPseudostateCompositeStates}
               {Initial pseudostates with composite states.}


\subsection{Regions}

As discussed in~\S\ref{ss:states}, LogixUML supports only simple and
composite states. Simple states have no regions, and composite states
can have only a single region containing the initial pseudostate and
all substates.

A region can be activated in two ways per \textcite[\S14.2.3.2]{OMGUML}:
\introTechnicalTerm{default activation} or
\introTechnicalTerm{explicit activation}.
Figure~\ref{fig:regionDefaultActivation} shows the two cases where
default activation is applicable due to a transition terminating
on the boundry of a composite state. The transition triggered by
event \identifier{e1} will complete with state \identifier{super1}
active, but will not enter state \identifier{sub1}.
Since composite state \identifier{super2} includes an initial
pseudostate, the transition triggered by event \identifier{e2} will
result in entering both \identifier{super2} and \identifier{sub2}.

\plantUmlFigure{regionDefaultActivation}
               {Entering regions via default activation.}

An example of explicit activation is depicted in
Figure~\ref{fig:regionExplicitActivation}, where the transition triggered
by event \identifier{e} terminates on the contained substate
\identifier{sub1}. In this case, states \identifier{super1} and
\identifier{sub1} are entered;
state \identifier{sub2} is not entered because the initial pseudostate
applies only when the region is entered with default activation per
\textcite[\S14.2.3.7]{OMGUML}.

\plantUmlFigure{regionExplicitActivation}
               {Entering a region with explicit activation.}


\subsection{Transitions}

Three transition types are defined in \textcite[\S14.2.3.8.1]{OMGUML}:
\introTechnicalTerm{external}, \introTechnicalTerm{local},
and \introTechnicalTerm{internal}. Only external and local transitions
are supported with the exception that self-transitions are not allowed; the
source and target states must be different.

The method by which LogixUML distinguishes between external and local
transitions differs from the standard. \textcite[\S14.2.4.9]{OMGUML}
states a transition is external if the arrow departs the source's
boundry, or local if the arrow is fully-contained within the source's boundry.
LogixUML only evaluates the relationship between the source and target states;
how the transition arrow is drawn makes no difference. The transition is
treated as local if the target is contained within the source, otherwise
it is considered external.

Figure~\ref{fig:localTransition} shows two transitions, both
treated as local. The transition to \identifier{sub1} is drawn
as a traditional local transition with the arrow completely within the
source. LogixUML executes the second transition to \identifier{sub2} in the
same manner, even though it is drawn as an external transition per the
standard. Neither transition will trigger the exit action of their
respective source states.

\plantUmlFigure{localTransition}
               {Examples of local transitions.}

It is recommended to draw all local transitions using the standard method,
regardless of the fact that LogixUML does not use the arrow path
to differentiate between external and local transitions. Maintaining
consistency between the visual depiction and system behavior
reduces confusion by not unnecessarily deviating from established
notation.

The duration of a transtion, measured in the number of times the state machine
add-on instruction must be consecutively scanned true, is nonzero for all
transitions. Exactly how many scans are required is configurable,
and is discussed in detail in~\S\ref{sec:scanMode}.


\subsection{Events}

The implementation of events is based on \introTechnicalTerm{ChangeEvents}
as described in \textcite[\S13.3.3.3]{OMGUML}, with a boolean AOI input
serving as the \introTechnicalTerm{changeExpression}. LogixUML extracts
the string entered in the \inputField{Received event} property for
all transitions
in the state machine, and defines a boolean input parameter for each,
adding the \identifier{event\_} prefix. Since events are used to create
input parameters, their names must be valid add-on instruction parameter
names.

Event inputs are edge-sensitive; the add-on instruction interprets the
input rising edge as an event occurrence. Furthermore,
the rising edge must occur after the AOI is initially scanned true.
The timing diagram shown in Figure~\ref{fig:eventFirstScan} illustrates
event input edge detection. The AOI is first scanned true at scan~1, and
energzies its \identifier{active} output; this can be a result of either
the processor's first scan or if the AOI has conditional rung-in
conditions which transition from false to true.
The input for event \identifier{e1} also transitions true at scan~1, but
is ignored as it did not occur after the AOI's initial scan true.
Scan~2 presents the AOI with a rising edge for event \identifier{e2},
which is interpreted as an event occurrence and handled accordingly.

\plantUmlFigure{eventFirstScan}
               {Event inputs at first scan.}

A simple method for generating an event for conditions that may be
true when the AOI is first scanned true is to use the AOI's
\identifier{active} output as a series condition for energizing the
event's input. The \identifier{active} output is unconditionally energized
when the AOI is first scanned, which will generate a rising edge on the
next scan.

Events can be reused to trigger multiple transitions by using the
same name as the \inputField{Received event} for several transitions.
The only restriction is the same event cannot trigger more than
one transition originating from a given state.
Event names are not case-sensitive as they are used to generate AOI input
parameters, which are also case-insensitive, so events that differ only in
case will be considered the same.

The transitions shown in Figure~\ref{fig:duplicateEvent} provide an example
of event reuse, all of which will yield two AOI input parameters:
\identifier{event\_x} and \identifier{event\_y}.
The transitions originating from state \identifier{s1} are invalid because
both are triggered from the same event, \identifier{y},
resulting in ambiguity in determining which to execute in response to the event.
The transitions triggered by event \identifier{x} are valid, even though both
are enabled in the context of state \identifier{sub}.
Transitions originating from a substate have priority over those
from a containing superstate per \textcite[\S14.2.3.9.4]{OMGUML}, so
event \identifier{x} would cause a transition to \identifier{s2} if the
state machine is in \identifier{sub}, or to \identifier{s1} if the state
machine is in \identifier{super} but not \identifier{sub}.

\plantUmlFigure{duplicateEvent}
               {Duplicate event names.}

Modelio's default theme settings do not show a transition's triggering
event in the diagram. Event names can be made visible by selecting
\menu{Configuration > Diagram themes}, and enabling the
\directory{Transition / Show label} option.


\subsection{Stereotypes}

Only the stereotype defined by the LogixUML module,
\identifier{\stereotypeName}, is permitted, and it must be applied to
every state machine that will be exported to an add-on instruction.
See \S\ref{??} for additional details.


\subsection{Notes}

Note elements may be attached to any element. Their presence and content
are ignored by LogixUML.
