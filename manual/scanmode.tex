\section{Transition Scan Mode}
\label{sec:scanMode}

A state machine add-on instruction effects transitions by actuating its
do, exit, and entry outputs throughout a series of sequential scans.
Each time the AOI is scanned during a transition these outputs are updated
to reflect the next step in the transition until the target state is achieved.

The number of scans required to complete a transition and the outputs
energized in each scan are configured with the
\identifier{transitionScanMode} property of the \identifier{\stereotypeName}
stereotype applied to the state machine. All transitions within the state
machine operate according to the selected scan mode, although different
state machines within the same Modelio project may utilize different
transition scan modes.

Three transition scan modes are available: single, dual, and sequential.
The \identifier{transitionScanMode} property has no default value, and
must be explicitly set to one of these values. The selection is not
case-sensitive, and any surrounding whitespace is ignored.

The transition shown in Figure~\ref{fig:transitionScanModeStateMachine}
will be used to describe the behavior of each transition scan mode.

\plantUmlFigure{transitionScanModeStateMachine}
               {Transition scan mode example state machine.}

Regardless of the selected scan mode each exit and entry output involved in
a transition is energized only once during a given transition, and only for
a single scan. The scan modes differ in the set of outputs activated in
each transitional scan.


\subsection{Single Mode}
\label{ss:scanModeSingle}

When \identifier{transitionScanMode} is set to single the entire transition,
regardless of topology, is performed in one scan. All exit and entry
outputs are energized simultaneously, followed by the stable condition
of the target state.

Figure~\ref{fig:transitionScanModeSingleTiming} illustrates single
transition scan mode behavior.
The triggering event is delivered to the state machine at scan~1; all do
outputs are de-energized, and all exit and entry outputs are energized.
The transition is complete at scan~2 with all entry and exit outputs
turned off and the do outputs for both \identifier{super2} and
\identifier{target} energized.

\plantUmlFigure{transitionScanModeSingleTiming}
               {Single transition scan mode timing example.}


\subsection{Dual Mode}

Dual mode separates the exit and entry outputs into distinct scans;
exit outputs are energized first, followed by entry outputs.
For most transitions this will occupy two scans, however, some transitions
do not incur both exit and entry actions. The most common example is
the top-level initial transition for the entire state machine, which never
leaves any state. Transitions that only exit or enter states,
but not both, will complete in one scan in dual mode.

A timing chart for the example transition running in dual mode is shown
in Figure~\ref{fig:transitionScanModeDualTiming}. The triggering event
initiates the transition at scan~1, de-energizing all do outputs and turning
on all exit outputs. The entry outputs are then activated at scan~2, and
the transition completes in scan~3.

\plantUmlFigure{transitionScanModeDualTiming}
               {Dual transition scan mode timing example.}


\subsection{Sequential Mode}

The sequential scan mode allocates a scan for each state boundry that must
be traversed, energizing a single exit or entry output each
scan. This sequence begins with exiting the source state. If the source
is contained in a composite state, subsequent scans will progressively
leave the next immediate composite state. Each scan that exits a state will
cause two transitions of the outputs associated with the state being
exited: the do output will deenergize and the exit output will energize.
Exit scans will continue until the common composite state containing both
the source and target is reached, if one exists.

The transition will begin entering states after exiting all states that need
to be left, starting with the outermost composite state, and
continuing inwards until arriving at the ultimate target. Each state that
needs to be entered requires two scans. The first causes the entry output
to energize, and the second turns the entry output off and the do output on;
the do output remains on for the remainder of the transition.

The timing chart in Figure~\ref{fig:transitionScanModeSequentialTiming}
shows how the example transition operates in sequential mode.
When the triggering event arrives at scan~1 it causes an exit from only the
innermost origin state, \identifier{source}. Egress continues in scan~2
by exiting the next level composite state, \identifier{super1},
in the same fashion. All states that need to be exited have done so at scan~3,
and the transition begins entering destination states starting from
\identifier{super2}. The composite state \identifier{super2} has been entered
by scan~4 and the transition proceeds to enter \identifier{target}.
The transition is then complete at scan~5 where \identifier{super2} and
\identifier{target} have been fully-entered and the stable condition has
been attained with both do outputs on.

\plantUmlFigure{transitionScanModeSequentialTiming}
               {Sequential transition scan mode timing example.}

Unlike single and dual mode, the number of scans required to complete a
transition can vary in sequential mode. It is still deterministic in that
the number of scans for a given transition is a function of the state
machine topology, so the same transition will always require the same number
of scans. However, other transitions can differ, as opposed to single and
dual mode where the maximum number of scans for all transitions is fixed,
and independent of state machine configuration.
